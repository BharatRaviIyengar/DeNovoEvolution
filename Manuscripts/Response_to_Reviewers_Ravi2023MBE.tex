\documentclass[12pt,a4paper]{article}
\usepackage[margin=2.5cm]{geometry}
\usepackage{amsmath,amssymb,graphicx,setspace,palatino,float,subcaption, booktabs, titlesec}
\usepackage[colorlinks, allcolors = blue]{hyperref}
\usepackage{natbib}
\citestyle{egu}

\titlespacing*{\paragraph}{0pt}{1\baselineskip}{3pt}
\setlength{\parindent}{0pt}
\setlength{\parskip}{\baselineskip}

\begin{document}

\onehalfspacing 

\section*{Editor}

Dear Editor,

Enclosed please find a major revision of manuscript number MBE-23-0015, titled ``Neutral models of \textit{de novo} gene emergence suggest that gene evolution has a preferred trajectory''. We are very grateful to you for co-ordinating the review process. We are also thankful to both the reviewers for their detailed analysis of our manuscript, which allowed us to improve our own understanding of our results, and helped us to improve our work. We were able to adddress the reviewers' concerns by performing additional analyses of our data, as well as by revising our manuscript. 

Before embarking on a point-by-point reply to the reviewers' comments, we summarize the major structural changes we have made in our manuscript:

\begin{itemize}
\item We have revised our model based on several suggestions by the reviewers. We include the results from our older model in the supplementary material, for the reader's interest. 
\item We have changed figure 1 to a three panel figure such that it shows probabilities of fixation under different selection conditions.
\item In response to the reviewers' comments, we have also made several changes to the main text. We have highlighted these changes using a different font color.
\end{itemize}

We hope that this revision addresses all the concerns and makes our manuscript acceptable for publication. Thank you very much again for considering this manuscript.

Yours sincerely,
Bharat Ravi and Erich Bornberg-Bauer

\section*{Reviewer: 1}

{\itshape Comments to the Author

One of the most valuable points of this paper is that it formalizes many principles that are intuitive, but which have not been explicitly discussed in the field of de novo gene emergence with theoretical rigor. The authors build a simple probabilistic model for the neutral evolution of de novo protein coding genes, which I think is important because it could inform about what to expect under a neutral scenario given the molecular knowledge we have gathered over the past 60 years.

Because of its theoretical approach this paper is a good addition to the de novo gene literature, however, I think the paper will be more relevant if it would either:

\begin{enumerate}
\item offer a more exhaustive contextualization of the model and the field, emphasizing why the model is showing results that cannot be addressed with empirical data, or

\item it would evaluate some of their results with empirical genetic diversity or experimental evolution datasets (such as https://academic.oup.com/mbe/article/39/2/msab368/6491261) to test whether the rates of ORF gain or loss within a population match their predictions. As it stands, although formalizing these probabilities theoretically has the potential to be an important contribution, given its simplifications, I believe the model is not too informative of the realistic potential of its predictions.
\end{enumerate}
}

We thank the reviewer for both suggestions, and have tried to address both. We have now expanded our discussion to speculate why empirical data may be different from the model's predictions (lines 528 -- 535, 556 -- 566). Specifically, we note that real parameters may be different from the ones estimated from currently available data. However, we also note that many predictions are not qualitatively affected by small changes in parameter values. Furthermore, we now suggest experiments that can provide data for better theoretical predictions (lines 5616 -- 566)

As reviewer suggested we also compared our predictions with existing data. A recent study (yet unpublished in a peer-reviewed journal), uses coalescence theory to estimate the rates of transcript gain and loss in geographically isolated populations of \textit{Drosophila melanogaster} (\href{https://doi.org/10.1101/2023.02.13.528330}{Grandchamp et al. 2023}). We compared the estimates of this study with ours and show that the two estimates do not differ by a large margin (lines 237 -- 246). Furthermore, we reason why the differences could occur (576 -- 585). We now cite the study mentioned by the reviewer (\href{https://doi.org/10.1093/molbev/msab368}{Lange et al., 2021}) to reason how the real effective population size could be different from the previous estimates, and that environmental conditions play a role in gene evolution. However, we haven't used this study for calculation of gene gain and loss rates because it does not explicitly report the evolution of new genes.


{\itshape When it comes to the model itself, I had a problem with some limitations concerning basic molecular biology processes and biases that are relevant for the origination of de novo genes. While some of these points were touched upon in the discussion, I still think that their model could easily incorporate ways of addressing them, which would make this model much more realistic and relevant. I specifically find four points to be discussed in more detail or incorporated into the model:



\textbf{1) Basal transcription:} The authors repeatedly mention that in order for transcription to take place there is a need for a promoter to evolve. Much empirical work in the field of de novo gene emergence has shown how basal levels of transcription or expression due to chromatin rearrangements might prime the functionalization of de novo genes 

\begin{itemize}\setlength{\itemsep}{-5pt}
\item https://genome.cshlp.org/content/20/10/1313
\item https://elifesciences.org/articles/47138
\item https://academic.oup.com/mbe/article/37/4/1165/5679780
\item https://genome.cshlp.org/content/28/11/1675.short
\item https://elifesciences.org/articles/09977
\end{itemize}

Because the presence/absence of transcription is a fundamental aspect of the model in this paper, I think this paradigm needs to be integrated into it for this study to be relevant given the knowledge we now have.
}



We thank the reviewer for this suggestion and for pointing us to relevant studies. It is indeed true that low level of transcription exists even in the absence of a core promoter motif. Furthermore enhancers can also initiate transcription in their proximal regions. However, we also note that vast majority of RNAs arising due to pervasive transcription (including the enhancer associated eRNAs), are degraded in the nucleus and thus do not reach the cytoplasm where they can be translated. Therefore, we think that a polyadenylation signal would still be a necessary feature for determining transcript length and stability. We further rationalize this argument by two observations. First, polyA signals are 10-fold less enriched in coding regions (0.0004) than in intergenic regions (0.004). PolyA signals are also 3 times less enriched in \textit{D.melanogaster} RNAs than in the intergenic regions. Furthermore, the polyA signals in RNAs are usually located at the end of the RNA. These observations suggests that polyA signals within the gene body (including ORF) can be disruptive. More importantly, many studies on \textit{de novo} gene discovery to our knowledge, employ a RNAseq library preparation kit that amplifies polyadenylated RNAs (\href{https://doi.org/10.7554/eLife.09977}{Neme and Tautz, 2016}; \href{https://doi.org/10.7554/eLife.47138}{Witt \textit{et al.}, 2019}; \href{https://doi.org/10.1093/molbev/msz300}{Majic and Payne, 2019}; \href{https://doi.org/10.1038/s41467-021-20911-3}{Blevins \textit{et al.}, 2021}). Thus most of the discovered \textit{de novo} genes indeed have a polyA tail (and must also have a polyadenylation signal). We now mention this observations in the main text (lines 99 -- 104).

We next focus on sequences that initiate transcription. Although several DNA sequence motifs corresponding to enhancers and core promoters have been identified, there is no set of sequence motifs that is deemed to be absolutely necessary for this process. The best understood, and the most common core promoter motifs are TATA-box and Inr, which we had included in our original analysis. We further note that although some sequences are indeed associated with enhancers, the actual functional enhancers are largely determined by epigenetic marks that vary significantly in different cell types, developmental stages, and during different environmental conditions. One common feature of both promoters and enhancers is that they both are characterized by a nucleosome depleted region (\href{https://doi.org/10.1038/s41576-019-0173-8}{Andersson and Sandelin, 2020}). Overall, we surmise that transcription can be initiated at any nucleosome depleted region but the stability of such transcripts is determined by polyA signals. We now explain these arguments in lines 95 -- 105.

We assume that such nucleosome free regions are uniformly distributed in open chromatin and these regions can initiate transcription at a basal rate. To this end, we first estimated the intergenic genome of \textit{D. melanogaster} that are in an open chromatin state. Next, we calculated the fraction of known cis-regulatory elements (including enhancer) that are present in the open intergenic chromatin regions. This fraction is our estimate of basal rate of transcription initiation. Using this estimate, we now calculated the probability of transcript emergence based on poly-A signals alone, while ignoring the requirement of a core promoter. We first asked if such an assumption would significantly change the results presented in our first draft (Figures 1 -- 3). We found that these results do not change significantly. The fundamental inferences remain the same (for example, compare figures 1 and S1). We now mention the similarities and differences between the promoterless (new) and with-promoter (old) models in the main text (lines 234 -- 236, 276 -- 278, 308 -- 310). If we had data on every promoter/enhancer sequence, and if we incorporate them in the model, then the exact RNA probability values should lie in between those estimated by the promoterless model and TATA/Inr-only model. Therefore, in the revised manuscript we now present the results from the promoterless model, and provide a suitable justification in the text (lines 123 -- 126) including a new materials and methods section on how we inferred the general likelihood of transcription initiation (714 -- 733). We include the old results based on TATA/Inr-only model in the supplementary data (Fig S1 -- S3). 

{\itshape \textbf{2) Non-neutral background:} de novo ORFs might emerge in non-coding regions that still encode for important cellular functions, such as introns or enhancers. This would mean that some mutations that lead to the functionalization of a de novo peptide might be unviable because they temper with the functions of the genomic regions where they arise. This might make some sequences harder or easier to evolve than others because some specific genomic positions will be locked in to a given nucleotide or arrangement of nucleotide. The way this could be integrated into the model is by considering what is the percentage of sites that tends to be strongly constrained or conserved between different species.}

The reviewer raised an important concern. To address this concern, we performed an analysis of total number of available sites and the probability of gene emergence, focusing on the model organism \textit{D. melanogaster}. To this end, we first downloaded the genomic coordinates of all \textit{cis}-regulatory motifs from \href{http://redfly.ccr.buffalo.edu/}{REDfly database} (please also see our reply to the previous comment). Next, we calculated the lengths of intergenic sequences that are not interrupted by any \textit{cis}-regulatory motif. This would constitute the genomic regions where new genes are free to emerge. We also estimated the total number of sites available for genes of different lengths to emerge. Finally, we calculated the total gene gain probability by multiplying the total number of feasible sites and per-site gene gain probability. Using the same principle, we calculated the rate of transcript gain that we compare with the population data from \href{https://doi.org/10.1101/2023.02.13.528330}{Grandchamp et al. 2023}.  We now briefly describe these findings in the main text (lines 237 -- 246, 577 -- 585).

We would like to note that in the original analysis, we calculate the probabilities of gene loss and gain, at one genomic locus. We have now explicitly mentioned this in the revised manuscript (line 123).  We also assume that this locus exists in an intergenic region devoid of any functional sequences. Therefore the original results would not be dependent on the number of feasible loci. 

{\itshape \textbf{3) Structured probabilities:} The authors rightly used empirical estimations of CG content to study the probability of functionalizing or defunctionalizing mutations. But It's not just CG content, it might be important to consider the frequency or presence/absence of genomic patterns like CpG islands or microsatellites. The existence of these features will strongly influence the likelihood of getting specific kinds of mutations, which will be important for the biases in evolution as well as the propensities in protein evolution that the authors tried to study with their model. Not to mention that these features will have different probabilities of evolving transcription on top of the coding sequence.}



We thank the reviewer for this valuable suggestion. We agree with them that nucleotide distribution is not uniform and therefore the probabilities of different DNA sequences cannot be estimated accurately from GC content alone. We also note that sequence composition can be highly variable between different species, as well as between different genomic regions in the same species. Thus the most accurate way to estimate the rate of gene evolutionary dynamics would be to analyse each genomic locus individually. Although such an analysis would be accurate, it would be difficult to generalize. To keep a balance between accuracy and generalizability of our results, we repeated our original calculations of various probability using nucleotide distribution estimates from intergenic regions of \textit{Drosophila melanogaster}  (flybase release 6.49). Specifically, we estimated the frequencies of all hexamers in the intergenic sequences using a sliding window. We ignored the hexamers that contained an ambiguous nucleotide character (non ATGC). From the frequency distribution of all the 4096 hexamers, we estimated the probability of finding (also gaining and losing) the polyadenylation signal sequences (that are 6 nucleotides long). Similarly, we calculated the probabilities of start and stop codons, using a distribution of trimers estimated from the \textit{D.melanogaster} intergenic regions (lines 604 -- 607). With these probabilities, we repeated the analyses shown in the original figures 1A, 2A, 2C and 3A. 

We found that the the results from actual nucleotide distributions indeed differed from the ones obtained using total GC content of \textit{D.melanogaster} intergenic genome (0.41). We now include these results in the main manuscript with the predictions using GC-content. We also explain the details of the re-analysis and its rationale (as described in this response), at several places in the main text (lines 197 -- 206, 600 -- 603). We also use the oligomer based nucleotide frequency distributions to calculate the fixation probabilities and rate of gene/transcript gain.



{\itshape \textbf{4) Population genetics:} Ultimately a de novo gene needs to increase in frequency in a population if it is going to be maintained in the genome. The probability of fixation of a neutral allele is 1/2N, but how is the birth/death mutational dynamics of de novo genes that the authors are exploring affected by the demography of species? I believe that drift might be the primary source of loss of de novo neutral genes, much more than mutational defunctionalization. Or do the authors want to disregard this and just focus on the macroevolutionary retention of de novo genes in phylogenies? If that is the case, this could be discussed, maybe not necessarily incorporated into the model.}

This is indeed a very useful suggestion. We now calculate the probability of fixation as well as the probability of extinction under the assumptions of (1) total neutrality, (2) weak positive selection, and (3) weak negative selection. We include these analyses in the new figure 1, and describe the rationale (lines 207 -- 232) and the method (lines 800 -- 802, new table 2) in the main text.

{\itshape I think that if the authors would follow at least some of these recommendations then they would be walking on much more solid ground and they would not need to use the ``all models are wrong'' quote, in favor of a more grounded implications of their findings. This would also mean that we would count with a much more solid point of argument for the discussion of the likelihood of de novo gene emergence, a field which tends to be too empirical and with little theory, which is why this paper would be a good addition to the literature if improvements are made.}

We thank the reviewer again for all their comments, and hope that we have addressed all their concerns satisfactorily.

\section*{Reviewer: 2}

{\itshape Comments to the Author
This paper describes a model to calculate the mutational probabilities of de novo gene origin and loss. Its conclusions are that certain modes of gene origin are more probable than others. I am no expert in these kinds of models or in de novo gene emergence, but I found these arguments interesting and valuable.}

We thank the reviewer for their comments and suggestions.

{\itshape I have two major comments about the paper. The first is that I was surprised to see no arguments or formalism rooted in population genetics in this paper. The authors only calculate mutational probabilities, but no fixation probabilities. They state several times that they work under the assumption of total neutrality, so I guess an informed reader can put two and two together, knowing that in Kimura's model neutral fixation probabilities only depend on the mutation rate of neutral alleles. I think the paper would benefit from making this explicit -- the assumption of neutrality justifies only caring about mutational probabilities. Of course, the neutral model comes with its own assumptions (diploid organisms  etc). Perhaps this could be acknowledged somewhere. And while fixation probabilities only depend on the mutation rate, I think the time to fixation under neutrality does depend on the effective population size. So the authors' model as it stands does not make predictions about how long it takes for de novo genes to fix or be lost (though this extension may be easy to carry out).}

We thank the reviewer for this important suggestion. We now calculate the probability of fixation as well as the probability of extinction under the assumptions of (1) total neutrality, (2) weak positive selection, and (3) weak negative selection. We include these analyses in the new figure 1, and describe the rationale (lines 207 -- 232) and the method (lines 796 -- 798, new table 2) in the main text. 

We now explicitly state that fixation under total neutrality depends solely on the mutation rate (lines 221 -- 222). We have also mentioned that time to fixation would be much smaller than time to gain a gene (lines 230 -- 232)

{\itshape Another difficulty for me was the argument about whether the existence of a lineage specific gene really implies it evolved only in that lineage. The alternative being an earlier gain with losses in all lineages but one. The paper justifies this by deriving that the probability of losing a de novo gene is much larger than gaining it. I struggle to see how that comparison alone is sufficient to make this argument. It seems to me one would have to tally up all relevant probabilities of gaining, losing, and keeping the gene along the relevant internal and terminal branches of the species tree. The length of these branches also seems like it might matter a lot. A gain-with-2-losses trajectory, like the authors suggest may be likely, requires the new gene to be gained and then kept along one internal and one terminal branch. It would seem to me that this is vastly less likely under strict neutrality than a lineage specific gain. I may of course be wrong, but I think a very simplistic argument based on only two probabilities with no treatment of the actual phylogeny is not sufficient to convince me of the authors' argument.}

We are not sure if we understood the reviewer's point completely. We do agree that phylogenetic inferences should take into consideration branch lengths. We assume that branch lengths should have minimal effect when comparing gene gains and losses in populations that haven't diverged from each other over a long time. We now explicitly say that we analyse populations (lines 148 -- 154) and not species that could have diverged over many millions of years. Thus our model is suitable for relatively shorter timescales. Furthermore, multiple gene gain and loss events within a lineage or branch would be highly rare, given the low probability of these events. 

Thus given an observation that a population has a gene but two other populations do not have it, there can be two assumptions:

1. Gene is gained in one population
2. Gene was present in the ancestor and lost in two populations

The parsimonious assumption would be that gene is only gained in one population. What we show is that the gene \textbf{can} be lost in two populations independently. It is indeed possible that these two populations in turn share a common ancestor that had already lost the gene. This event is more probable than two gene losses. Multiple independent losses is an extreme case that may be less feasible than a single loss followed by population divergence. Nevertheless it is possible, and can be more likely than a single gene gain event. We hope we have clarified the reviewer's concern.

{\itshape Another minor comment from me is that in the discussion it might be valuable to address that at least some de novo genes become essential quickly (I would for example point to work by the Long lab, which is cited already). This would allow purifying selection to quickly preserve such genes against losses.}

We thank the reviewer for their suggestion and cite the relevant work in lines 470 -- 471.

{\itshape Lastly, I couldn't find the actual parameter values used for the various probabilities in the paper. I know the authors will make the code available, but for the casual reader a table would be valuable.}

Our primary parameters are mutation rate and mutation rate bias (shown in Table 1). All other parameters are derived and we therefore do not mention their values explicitly. Furthermore, the derived parameters are complex functions of gene length and nucleotide composition, and therefore, vary significantly. We now include the frequencies of trimers and hexamers from \textit{D. melanogaster} intergenic regions as supplementary files. 

\end{document}
